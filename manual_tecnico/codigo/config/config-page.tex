\documentclass[oneside,12pt]{article}

%Bibliotecas
\usepackage[spanish, mexico]{babel}                 %Documento en español
\usepackage[utf8]{inputenc}                         %Codificación en utf8
%\usepackage[sfdefault]{ClearSans}                  %Fuente sin serifa
\usepackage[letterpaper, headheight=1.0in,
            textwidth=6.5in, inner=1.0in,
			outer=1.0in, bottom=1.0in]
            {geometry}                              %Margenes del documento
\usepackage[breaklinks=true]{hyperref}
\usepackage{graphicx}                               %Manejo de imágenes
\usepackage{float}                                  %Portada
\usepackage{fancyhdr}                               %Encabezado y pie de página
\usepackage{xcolor}                                 %Gestión de colores
\usepackage[]{hyperref}                             %Opciones de pdf producto
\usepackage{enumerate}
\usepackage[T1]{fontenc}
\usepackage{times}
\usepackage{color}
\usepackage{listings}
\usepackage{amssymb}
\usepackage{amsmath}

\definecolor{dkgreen}{rgb}{0,0.6,0}
\definecolor{gray}{rgb}{0.5,0.5,0.5}
\definecolor{mauve}{rgb}{0.58,0,0.82}

%Formato de párrafos
\setlength{\parindent}{2em}                         %Sangría
\setlength{\parskip}{0em}                           %Espacio entre párrafos
\renewcommand{\baselinestretch}{1.5}

%Créditos
\title{Practica Número 02}
\author{Rafael Landa Aguirre}

%Encabezado y Pie
\pagestyle{fancy}
\fancyhf{}
\chead{Práctica 02. Rafael Landa Aguirre.}
\rfoot{Página \thepage}
\lfoot{Compiladores}
\renewcommand{\headrulewidth}{1pt}
\renewcommand{\footrulewidth}{1pt}

\lstset{
  frame=tb,
  language=Java,
  aboveskip=3mm,
  belowskip=3mm,
  showstringspaces=false,
  columns=flexible,
  basicstyle={\small\ttfamily},
  numbers=none,
  numberstyle=\tiny\color{gray},
  keywordstyle=\color{blue},
  commentstyle=\color{dkgreen},
  stringstyle=\color{mauve},
  breaklines=true,
  breakatwhitespace=true,
  tabsize=3
}

% minimizar fragmentado de listados
\lstnewenvironment{listing}[1][]
{\lstset{#1}\pagebreak[0]}{\pagebreak[0]}

\lstdefinestyle{consola}
{
	basicstyle=\scriptsize\bf\ttfamily,
	backgroundcolor=\color{gray75},
}

\lstset{ language=Java }

%Configuración de pdf
\hypersetup{
    pdftitle={Practica 02 Mostrador de 
    	figuras en Java AWT y YACC},
    pdfauthor={Rafael Landa Aguirre},
    pdfsubject={Practica 02 - Compiladores 
    	- Mostrador de figuras en Java AWT y YACC},
    bookmarksnumbered=false,
    bookmarksopen=true,
    bookmarksopenlevel=3,
    pdfstartview=Fit,
    pdfpagemode=UseOutlines,
    pdfpagelayout=TwoPageRight,
    colorlinks,
    linkcolor={black!255!black},
    citecolor={black!50!black},
    urlcolor={black!80!black}
}

%Espacios
\newcommand\tab[1][1cm]{\hspace*{#1}}