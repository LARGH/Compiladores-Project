\subsection{Java}
%% Describir Java
Java es un lenguaje de programación y una plataforma informática comercializada por primera vez en 1995 por Sun Microsystems. Hay muchas aplicaciones y sitios web que no funcionarán a menos que tenga Java instalado y cada día se crean más. Java es rápido, seguro y fiable. Desde portátiles hasta centros de datos, desde consolas para juegos hasta súper computadoras, desde teléfonos móviles hasta Internet, Java está en todas partes.\\

Java es la base para prácticamente todos los tipos de aplicaciones de red, además del estándar global para desarrollar y distribuir aplicaciones móviles y embebidas, juegos, contenido basado en web y software de empresa. Con más de 9 millones de desarrolladores en todo el mundo, Java le permite desarrollar, implementar y utilizar de forma eficaz interesantes aplicaciones y servicios. Desde portátiles hasta centros de datos, desde consolas para juegos hasta súper computadoras, desde teléfonos móviles hasta Internet, Java está en todas partes.\\

JDK\\

El JDK es un entorno de desarrollo para crear aplicaciones, applets y componentes utilizando el lenguaje de programación Java.
El JDK incluye herramientas útiles para desarrollar y probar programas escritos en el lenguaje de programación Java y que se ejecutan en la plataforma Java.\\

JRE\\

Java Runtime Environment (JRE) es lo que se obtiene al descargar el software de Java. JRE está formado por Java Virtual Machine (JVM), clases del núcleo de la plataforma Java y bibliotecas de la plataforma Java de soporte. JRE es la parte de tiempo de ejecución del software de Java, que es todo lo que necesita para ejecutarlo en el explorador web.\\
\subsection{Maven}
%% Describir Maven
Maven es una herramienta principalmente utilizada en el desarrollo de software Java. Aparece ante la necesidad de modelar el concepto de proyecto  y artefacto en forma estándar independendientemente del IDE de desarrollo.\\

Maven es una herramienta de software para la gestión y construcción de proyectos Java creada por Jason van Zyl, de Sonatype, en 2002. Es similar en funcionalidad a Apache Ant (y en menor medida a PEAR de PHP y CPAN de Perl), pero tiene un modelo de configuración de construcción más simple, basado en un formato XML. Estuvo integrado inicialmente dentro del proyecto Jakarta pero ahora ya es un proyecto de nivel superior de la Apache Software Foundation.\\

Maven utiliza un Project Object Model (POM) para describir el proyecto de software a construir, sus dependencias de otros módulos y componentes externos, y el orden de construcción de los elementos. Viene con objetivos predefinidos para realizar ciertas tareas claramente definidas, como la compilación del código y su empaquetado.\\

Se utiliza como herramienta de SCM (Software Configuration Management) para controlar:\\

\begin{itemize}
    \item El versionado de nuestros proyectos.
    \item Las dependencias con proyectos nuestros y librerías de terceros.
    \item Automatización de tareas relativas al ambiente de desarrollo y construcción del artefacto (compilación, generación de código, empaquetado, etc).
\end{itemize}