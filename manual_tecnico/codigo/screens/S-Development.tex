%Esta sección tratará acerca del desarrollo de la aplicación 
\subsection{Gramática del vector}
\label{gram:vector}
A continiación se presenta la gramática que describe la estructura del vector cuando el 
usuario introduzca información desde teclado:
\begin{itemize}
	\item instruccion:
	\begin{itemize}
		\item[$\rightarrow$] NUMERO 
		\item[$\rightarrow$] LINE NUMERO NUMERO NUMERO NUMERO
		\item[$\rightarrow$] CIRCLE NUMERO NUMERO NUMERO
		\item[$\rightarrow$] RECTANGLE NUMERO NUMERO NUMERO NUMERO 
		\item[$\rightarrow$] IMAGE NUMERO NUMERO
		\item[$\rightarrow$] COLOR NUMERO
				
	\end{itemize}
\end{itemize}

\subsection{Símbolos terminales usados en YACC}
Para poder describir la estructura de la gramática y traducirla a lenguaje real en
caracteres, tenenmos la siguiente información en el archivo \texttt{$vector.cal.y$}:

\begin{lstlisting}
%token NUMBER LINE CIRCLE RECTANGLE COLOR IMAGE PRINT
%start list
\end{lstlisting}

\subsection{Reglas gramaticales en YACC}
En el código siguiente se muestra las reglas gramaticales mostradas en la
sección \ref{gram:vector}, entonces, solo traducimos a código para YACC, 
no cambia mucho, solo hay que poner atención en los caracteres.
\begin{lstlisting}
%%
instr:  NUMBER  { /* constantes */
    			((Algo)$$.obj).inst = maq.code("constpush");
	                maq.code(((Algo)$1.obj).simb); 
		}
      | LINE NUMBER NUMBER NUMBER NUMBER { 
      	/* instrccion para dibujar lineas */
			maq.code("constpush");
                	maq.code(((Algo)$2.obj).simb); 
			maq.code("constpush");
                	maq.code(((Algo)$3.obj).simb); 
			maq.code("constpush");
                	maq.code(((Algo)$4.obj).simb); 
			maq.code("constpush");
                	maq.code(((Algo)$5.obj).simb); 
			maq.code("line");
		}
      | CIRCLE NUMBER NUMBER NUMBER { 
      	/* instrccion para dibujar circulos */
			maq.code("constpush");
                	maq.code(((Algo)$2.obj).simb);
			maq.code("constpush");
                	maq.code(((Algo)$3.obj).simb);
			maq.code("constpush");
                	maq.code(((Algo)$4.obj).simb);
			maq.code("circle");
		}
      | RECTANGLE NUMBER NUMBER NUMBER NUMBER { 
	   /* instrccion para dibujar rectangulos */
			maq.code("constpush");
                	maq.code(((Algo)$2.obj).simb); 
			maq.code("constpush");
                	maq.code(((Algo)$3.obj).simb); 
			maq.code("constpush");
                	maq.code(((Algo)$4.obj).simb); 
			maq.code("constpush");
                	maq.code(((Algo)$5.obj).simb); 
			maq.code("rectangulo");
		}
      | IMAGE NUMBER NUMBER { 
      	/* instrccion para colocar imagenes */
			Simbolo simbolo = new Simbolo();
			simbolo.setDibujable(new Imagen(
				cargarImagen("dalmata.jpg"),
				(int)((Algo)$2.obj).simb.val, 
				(int)((Algo)$3.obj).simb.val, f));
			maq.code("objectpush");
			maq.code(simbolo);
			maq.code("drawImage");
		}
      | COLOR NUMBER { 
      	/* instrccion para colorear contorno de las 
      	  figuras previas */
			maq.code("constpush");
	                maq.code(((Algo)$2.obj).simb); 
			maq.code("color");
		}
      ;
%%
\end{lstlisting}

\subsection{Analizador léxico en YACC}
El código siguiente se muestra cómo se analiza caractér por caractér 
cuando el usuario introduce información desde teclado. Primero, para poder
ejecutar la función \texttt{yylex()}, se tiene que haber invocado la función
 \texttt{yyparse()}.
\begin{lstlisting}
int yylex() {
	String s; //Cadena del usuario
	int tok;
	Double d;
	Simbolo simbo; //Cimbolos gramaticales
	if (!st.hasMoreTokens()) //Si no existen mas tokens
	      if (!newline) {
        	 newline = true; 
		 return ';';  //retornar el token
	      }
	else
	      return 0;
	s = st.nextToken();
	try { //Agregar a la tabla de simbolos la constante
		d = Double.valueOf(s);
		yylval = new ParserVal(
			new Algo(tabla.install("", 
				NUMBER, d.doubleValue()),0) 
		);
		tok = NUMBER; //retornar el lexema
	} catch (Exception e) {
	    /*Agregar a la tabla de simbolos el token
		LINE, CIRCLE, RECTANGLE, IMAGE, COLOR*/
		if (Character.isLetter(s.charAt(0))) {
			if((simbo = tabla.lookup(s)) == null)
        			yylval = new ParserVal(new Algo(simbo, 0));
		      	tok = simbo.tipo;
		} else {
			tok = s.charAt(0);
		}
	}
	return tok; // retornar el token detectado
}
\end{lstlisting}

\subsection{Almacenamiento simbolos gramaticales}
Para el almacenamiento de símbolos gramaticales se tiene planeada la 
elaboración de un objeto llamado Simbolo que permite definir una 
instrucción, el nombre, de que tipo es, la dirección de la primera
instrucción a la función que apunta. Además, la función a la que apunta 
deberá ser polifórmica de tipo Dibujable.

\begin{lstlisting}
class Simbolo {
	String nombre; //nombre del simbolo o variable
	short tipo;	// tipo de token
	double val; //valor en caso de ser constante
	Dibujable dibujable; // Objeto polimorfico. lineas, circulo, etc.
	String metodo; // Nombre del metodo a ejecutar
	int defn; // 
	Simbolo sig; //Apuntador al siguiente elemento de la lista
	Simbolo() { }
	Simbolo(String s, short t, double d) {
		nombre = s;
		tipo = t;
		val = d;
	}
    public Simbolo obtenSig() {
		return sig;
	}
    public void ponSig(Simbolo s) {
		sig = s;
	}
    public String obtenNombre() {
		return nombre;
	}
	public void setDibujable(Dibujable dibujable) {
		this.dibujable = dibujable;
	}
	public Dibujable getDibujable() {
		return dibujable;
	}
}
\end{lstlisting}

Para organizar la tabla de símbolos se tiene una lista simplemente
enlazada de objetos de tipo Simbolo.

\begin{lstlisting}
public class Tabla {
	Simbolo listaSimbolo;
	Tabla() {
		listaSimbolo = null;
	}
	Simbolo install(String s, short t, double d) {
		Simbolo simb = new Simbolo(s, t, d);
		simb.ponSig(listaSimbolo);
		listaSimbolo = simb;
		return simb;
	}
	Simbolo lookup(String s){
		for(Simbolo sp = listaSimbolo; 
		    sp != null; sp = sp.obtenSig())
			if((sp.obtenNombre()).equals(s))
				return sp;
		return null;
	}
}
\end{lstlisting}

\subsection{Implementación de operaciones vectoriales}
Se agregaron funciones para cada llamada a función tales como:
\begin{itemize}
\item \texttt{color()}.
\begin{lstlisting}
void color() {
	//Necesario para colorear contornos con AWT
   	Color colors[] = { Color.red, Color.green, Color.blue };
  	double d1 = ((Double)pila.pop()).doubleValue();
  	if (g != null) 
		g.setColor(colors[(int) d1]);
	return;
}
\end{lstlisting}


\item \texttt{line()}.
\begin{lstlisting}
void line() {
	//Necesario para insertar lineas con AWT
	double y_2 = ((Double)pila.pop()).doubleValue();
	double x_2 = ((Double)pila.pop()).doubleValue();
	double y_1 = ((Double)pila.pop()).doubleValue();
	double x_1 = ((Double)pila.pop()).doubleValue();
  	if (g != null) {
       		(new Linea((int) x_1, 
			   (int) y_1,
		  	   (int) x_2, 
		  	   (int) y_2))
			.dibujar(g);
	}
	return;
}
\end{lstlisting}

\item \texttt{circle()}.
\begin{lstlisting}
void circle() {
	//Necesario para insertar circulos con AWT
	double r  = ((Double)pila.pop()).doubleValue();
	double yc = ((Double)pila.pop()).doubleValue();
	double xc = ((Double)pila.pop()).doubleValue();
	Circulo circulo = new Circulo((int) xc, (int) yc, (int) r);
	circulo.transladar((-1)*((int) r), (-1)*((int) r));
  	if (g != null) {
       		circulo.dibujar(g);
	}
	return;
}
\end{lstlisting}

\item \texttt{drawImage()}.
\begin{lstlisting}
void drawImage() {
	//Necesario para insertar imagenes con AWT
	Simbolo simbolo = (Simbolo)pila.pop();
	Dibujable dibujable = simbolo.getDibujable();
	if ((g != null) && (dibujable != null))
		dibujable.dibujar(g);
	return;
}
\end{lstlisting}

\item \texttt{rectangulo()}.
\begin{lstlisting}
void rectangulo() {
	//Necesario para insertar rectangulos con AWT
	double y_2 = ((Double)pila.pop()).doubleValue();
	double x_2 = ((Double)pila.pop()).doubleValue();
	double y_1 = ((Double)pila.pop()).doubleValue();
	double x_1 = ((Double)pila.pop()).doubleValue();
  	if (g != null) {
     	(new Rectangulo((int) x_1, 
			   	(int) y_1,
		  	   	(int) x_2, 
	  		   	(int) y_2))
		.dibujar(g);
	}
	return;
}
\end{lstlisting}

\item \texttt{objectpush()}.
\begin{lstlisting}
void objectpush() { 
	//Necesario para insertar objetos en pila
		Simbolo s;
	  	s = (Simbolo) prog.elementAt(pc);
	  	pc = pc + 1;
	  	pila.push(s);
		return;
	}
\end{lstlisting}

\end{itemize}