%Esta sección está dedicada a la introducción
Logo es un lenguaje de programación de alto nivel, en parte funcional, en parte estructurado; de muy fácil aprendizaje, razón por la cual suele ser el lenguaje de programación preferido para trabajar con niños y jóvenes. Fue diseñado con fines didácticos por Danny Bobrow, Wally Feurzeig, Seymour Papert y Cynthia Solomon, los cuales se basaron en las características del lenguaje Lisp. Logo fue creado con la finalidad de usarlo para enseñar programación y puede usarse para enseñar la mayoría de los principales conceptos de la programación, ya que proporciona soporte para manejo de listas, archivos y entrada/salida. Logo cuenta con varias versiones.\\

Una característica más explotada de Logo es poder producir «gráficos tortuga», es decir, poder en dar instrucciones a una tortuga virtual, un cursor gráfico usado para crear dibujos, que en algunas versiones es un triángulo, en otras tiene la figura de una tortuga vista desde arriba. Esta tortuga o cursor se maneja mediante palabras que representan instrucciones.\\

El desarrollo de este proyecto consistirá en una interfaz gráfica para el usuario con un apartado en donde el pueda agregar código con la finalidad de dibujar desde las figuras más sencillas como cuadrados y círculos, hasta espirales, árboles , estrellas, etc, con la posibilidad de agregar colores a las figuras dibujadas, esto se mostrará en una vista en el mismo programa.\\

Para el desarrollo de la interfaz gráfica se cuenta con la API de JAVA: AWT, para 
aplicaciones de escritorio en Windows/Linux. \\